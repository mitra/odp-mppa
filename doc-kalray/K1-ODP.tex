\documentclass{trkalray}
\usepackage{listings}
\usepackage[toc,page]{appendix}
\usepackage{makeidx}
\usepackage{algorithm}
\usepackage{algorithmic}
\usepackage{caption}
\usepackage{graphicx}
\usepackage{listings}
\lstset{ %
frame=single
}

\newcommand{\MPPA}{MPPA\texttrademark\space}

\version{KETD-355}{W23}{2015}


\author{%
Kalray S.A.\autref{1}
}
\docowner{Nicolas Morey-Chaisemartin}{nmorey@kalray.eu}
\vers{1.1}
\institute{%
\autlabel{1} \email{support@kalray.eu},
Kalray S.A.
}

\abstract{%
This document describes how to use K1-ODP
}

\keywords{%
Dataflow, MPPAIPC, Examples
}

\renewcommand{\lstlistingname}{Code}% Listing -> Code


\makeindex

\title{K1-ODP Manual}

\begin{document}

\maketitle

\tableofcontents

\newpage
\section{Introduction}

This documents briefly describes how to use ODP for the MPPA platform.

ODP-MPPA is currently based on ODP 1.6.

Additional generic documentation can be found on ODP website and in
\begin{lstlisting}
/usr/local/k1tools/doc/ODP/mppa/
\end{lstlisting}

\section{Requirements}

Working with K1-ODP requires two sets of packages:
\begin{itemize}
\item[-]{A complete K1 Toolchain using the k1-tools package}
\item[-]{K1-ODP Library using the k1-odp package}
\end{itemize}

The K1-ODP files will be installed under the
\texttt{/usr/local/k1tools} directory.

\section{Choosing a version}
K1-ODP is available in multiple ports in the k1-odp package, one for
each of the platforms available for executing ODP applications.

In this version, two ports are available:
\begin{itemize}
\item[-]{\texttt{k1b-kalray-nodeos}}
\item[-]{\texttt{k1b-kalray-nodeos\_simu}}
\item[-]{\texttt{k1b-kalray-mos}}
\item[-]{\texttt{k1b-kalray-mos\_simu}}
\end{itemize}

\subsection{k1b-kalray-nodeos}

This port targets one Bostan MPPA cluster on hardware.

\subsection{k1b-kalray-mos}

Same target as \texttt{k1b-kalray-nodeos}, but running directly on
Kalray Hypervisor

\subsection{k1b-kalray-nodeos\_simu}

This ports targets one Bostan MPPA cluster in simulation.

Using the simulator, this ports allows simulated ODP applications to
have transparent access to the x86 network interfaces.

In the current version, this port supports these interfaces:

\subsection{k1b-kalray-mos\_simu}

Same target as \texttt{k1b-kalray-nodeos\_simu}, but running directly on
Kalray Hypervisor

\section{Packet IO Interfaces}

K1-ODP provides multiple pktio types that can be used to communicate
with Ethernet ports, or clusters.

\begin{itemize}
\item[-]{\texttt{loop}: Software loopback interface}
\item[-]{\texttt{e<X>}: 40G Ethernet interface of IOETH \texttt{<X>}}
\item[-]{\texttt{e<X>p<Y>}: 1/10G Ethernet interface \texttt{<Y>} of
  IOETH \texttt{<X>}}
\item[-]{\texttt{cluster<X>}: Interface to cluster \texttt{<X>}}
\item[-]{\texttt{magic:<if>}: Link to the x86 interface
  \texttt{<if>}. This is only available on simu platforms}
\end{itemize}

\subsection{Customizing Ethernet interface behaviour}

It is possible to configure the way ODP handles Ethernet interfaces.
This is done by passing options through the pktio name:
\begin{lstlisting}
e0p0:<option1>:<option2>
\end{lstlisting}

\subsubsection{Rx resources}

By default Ethernet packets from a lane are distributed among 20 NoC Rx on each
cluster that opens this lane.
When working with 40G lane, or with high PPS situations, this is often
not sufficient to handle the high performance requirement.
To configure this value, the option tags can be used:
\begin{lstlisting}
e0:tags=120
\end{lstlisting}

Not that it is not possuble to use more than 120 tags per pktio nor
250 tags for all pktios.

\subsubsection{Jumbo}

Jumbo frames are not allowed by default. To enable them, the option
\texttt{jumbo} must be used.
Note that the option must be the same (on or off) for all the cluster
that use an Ethernet interface

\begin{lstlisting}
e0:jumbo
\end{lstlisting}

\subsubsection{Loopback}

It is possible to put an Ethernet lane in loopback mode. In this mode,
it is not required for a cable to be plugged in, nor for the board to
have a physical connector connected to the lane.
When loopback mode is enabled, all the packets sent to this lane by
the clusters are looped back into the HW Dispatcher and sent back to
the cluster that opened this pktio.

Note that the option must be the same (on or off) for all the cluster
that use an Ethernet interface
\begin{lstlisting}
e0:loop
\end{lstlisting}

\section{Compiling and Running}

Some compiler flags must be passed to the compiler when trying to
build an ODP application.

\begin{lstlisting}
LDFLAGS += -L$($(K1_TOOLCHAIN_DIR)/lib/<port> -lodp
\end{lstlisting}
where port is replaced with one of the available K1-ODP port.

\section{Running}

Running K1-ODP applications is done the same way a standard K1 single
cluster application would be run.

\subsection{k1b-kalray-nodeos and k1b-kalray-mos}
Running a \texttt{k1b-kalray-nodeos} application, can be done either
on hardware using:
\begin{lstlisting}
k1-jtag-runner --exec-file "Cluster0:<executable name>" -- <args>
\end{lstlisting}
or in simulation using:
\begin{lstlisting}
 k1-cluster -- <executable name> <args>
\end{lstlisting}

\subsection{k1b-kalray-nodeos\_simu and k1b-kalray-mos\_simu}
Running a \texttt{k1b-kalray-nodeos\_simu} application, must be done in
simulation using this command:
\begin{lstlisting}
k1-cluster   --functional  --mboard=developer --march=bostan --user-syscall=/usr/local/k1tools/lib64/libodp_syscall.so -- <executable name> <args>
\end{lstlisting}

\section{Limitations}

\begin{itemize}
\item[-]{The current ODP version does not support ordered queues}
\end{itemize}

\end{document}
